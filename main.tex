\documentclass[11pt]{article}
\usepackage[utf8]{inputenc}
\usepackage{geometry}
\geometry{a4paper, margin = 1in}
\usepackage{titling}
\usepackage{amsmath, amssymb, mathtools, braket, physics}
\usepackage{graphicx} % Required for inserting images
\numberwithin{equation}{section}
\AtBeginDocument{\Large\selectfont}
\title{Derive Maxwell's equations from field theory viewpoint}
\author{YILUN LIN}
\date{June 2025}

\begin{document}

\maketitle
\textbf{This note aims to derive the Maxwell's equations in classical electrodynamics from the axiomatic field theory viewpoint. The derivation starts with the well-known expression of the Lagrangian $\mathcal{L}$ of a electromagnetic field in field theory and proceeds in three steps:}   \\
\hspace*{5em}{\textbf{1. Assumptions and Lagrangian;}}\\
\hspace*{5em}{\textbf{2. Variation\;$\longrightarrow$\; Euler-Lagrange Equations
\;$\longrightarrow$\; Inhomogeneous Maxwell's equations;}}\\
\hspace*{5em}{\textbf{3. Bianchi Identity of the Field Strength Tensor\;$\longrightarrow$\; Homogeneous Maxwell's equations.}} 
\\[1em]

\section{Lagrangian Density}
\textbf{(a) Basic field variable:}
The electromagnetic field is described by the four-potential $A_\mu(\Vec{x})$ in field theory.
\\[0.5em]
\textbf{(b) Definition of the strength of field:}
\begin{equation}\label{eq: def. field strength F}
   \mathbf F_{\mu\nu}\, \equiv \, \partial_\mu A_\nu-\partial_\nu A_\mu, \;\; 
   \mathbf F^{\mu\nu}\,=\,\mathbf{g^{\mu\alpha}g^{\nu\beta}F_{\alpha\beta}}.
\end{equation}
\textbf{(c) Construct the Lagrangian:}
\begin{equation}\label{eq:Lagrangian of electromagnetic field}
   \mathcal{L}\,=\,-\frac{1}{4} \mathbf{F_{\mu\nu}F^{\mu\nu}}\,-\,j^\mu A_\mu.
\end{equation}
\textbf{Note:}
\hspace{3em}\begin{enumerate}
\item The kinetic term $-\frac{1}{4}\mathbf{F_{\mu\nu}F^{\mu\nu}}$ ensures the correct field dynamics and energy-momentum tensor.\\
\item The second term in \eqref{eq:Lagrangian of electromagnetic field} enforces coupling to sources and yields charge conservation $\partial_\mu j^\mu\,=\,0$ by gauge invariance.
\end{enumerate}

\section{Variation and Inhomogeneous Maxwell's Equations}
It should be noted that the action
\begin{equation}
   \mathcal S\:=\:\int\! d^4x\, \mathcal{L}
\end{equation}
must be stationary on the boundary of the region.\\
\\[0.5em]
While the Euler-Lagrange equations read 
\begin{equation}
   \frac{\partial\mathcal{L}}{\partial A_\mu}\,-\,\partial_\nu\frac{\partial\mathcal{L}}{\partial(\partial_\nu A_\mu)}\,=\,0,
\end{equation}
thus
\begin{equation}
   \partial_\nu\frac{\partial\mathcal{L}}{\partial(\partial_\nu A_\mu)}\,\frac{\partial\mathcal{L}}{\partial A_\mu}\,=\,0.
\end{equation}
\\[0.5em]
Firstly, we compute
\begin{equation}
\begin{split}
\frac{\partial\mathcal{L}}{\partial(\partial_\nu A_\mu)}\, &= -\frac{1}{4}\cdot2\,
\mathbf F_{\rho\sigma}\frac{\partial \mathbf F^{\rho\sigma}}{\partial(\partial_\nu A_\mu)}\\ &=-\frac{1}{2}\,\mathbf F^{\rho\sigma}(\delta^\nu_\rho\delta^\mu_\sigma\,-\,\delta^\nu_\sigma\delta^\mu_\rho)\\
&=-\mathbf F^{\nu\mu},
\end{split}
\end{equation}
and since
\begin{equation}
\frac{\partial\mathcal{L}}{\partial A_\mu}\,=\,-j^\mu,
\end{equation}
substituting into the Euler-Lagrange equations gives
\begin{equation}
\begin{split}
\partial_\nu(-\mathbf F^{\nu\mu})\,-\,(-j^\mu)\,&=\,0\\
\partial_\nu\mathbf F^{\nu\mu}\,&=\,j^\mu.
\end{split}
\end{equation}
These are the inhomogeneous set of the Maxwell's equations\,(i.e.Gauss's law and the Ampere-Maxwell law):
\begin{equation}
    \nabla\cdot\mathbf E\,=\,\rho\;,
\end{equation}
\begin{equation}
    \nabla\times\mathbf B\:-\:\frac{\partial\mathbf E}{\partial t}\:=\:\mathbf j\;.
\end{equation}
\newpage

\section{Bianchi Identity and Homogeneous Maxwell's Equations}
Since $\mathbf F_{\mu\nu}$ is antisymmetric by definition \eqref{eq: def. field strength F}, we always have the identity
\begin{equation}
     \partial_\lambda\mathbf F_{\mu\nu}\,+\,\partial_\mu\mathbf F_{\nu\lambda}\,+\,\partial_\nu\mathbf F_{\lambda\mu}\,=\,0.
\end{equation}
This identity is exactly the homogeneous set of Maxwell's equations:
\begin{equation}
    \nabla\cdot\mathbf B\,=\,0\;,
\end{equation}
\begin{equation}
    \nabla\times\mathbf E\:+\:\frac{\partial\mathbf B}{\partial t}\:=\:0\;.
\end{equation}
\\[1em]
\textbf{We have already derived the classical Maxwell's equations from field theory viewpoint. }
\\[0.5em]
\begin{flushright}
    $\mathbf{QED.}$
\end{flushright}

\end{document}
